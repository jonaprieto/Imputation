\begin{Figure}
  \centering
\pgfplotstableread[col sep=comma,header=true, row sep=crcr]{
$table
}\data

  \begin{tikzpicture}
  \begin{axis}[
    axis lines = left
  , width  = \textwidth
  , height = 7cm
  % , ytick  = {0.76,0.78,0.80,0.82,0.84,0.86,0.88,0.90}
  % , ymin   = 0.75
  % , ymax   = 0.94
  , ylabel = {Accuracy Ratio (\%)}
  , grid   = major
  % , symbolic x coords
  %          = {1,2,3,4,5,6,7,8,9,10,11,12,13,14,15,16,17,18,19,20,21,22,23,24,25,26,27,28,29,30,31,32,33,34,35,36,37,38,39,40,41,42,43,44,45,46,47,48,49,50}
  % , xtick  = {1,2,3,4,5,6,7,8,9,10}
  , xlabel = {Sample of $dataset dataset}
  , enlarge x limits = 0.05
  , legend style     =
      { at     = {(0.5,1.25)}
      , anchor = north
      , legend columns = $nalgos
      }
  ]

  $addplots

  \legend{$algorithms}
  \end{axis}
  \end{tikzpicture}

  \fcaption{Behavior of the imputation algorithms with a missing values rate of $missingrate\% in the $dataset database. The vertical axis
  corresponds to the accuracy rate of the respective algorithm, and the
  horizontal axis represents the sample number in the experiment. We
  observe ARSI gets the best accuracy with respect the other algorithms in
  $arsiwins runs of a total of 50 runs.
  }
  \label{table:$missingrate-$dataset}
\end{Figure}
